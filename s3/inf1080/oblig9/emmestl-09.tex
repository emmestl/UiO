\documentclass[a4paper, norsk, 10pt]{article}

\usepackage[utf8]{inputenc}
\usepackage[a4]{}
\usepackage{amsmath}
\usepackage{amsthm}
\usepackage{amssymb}

\date{\today}
\title{Oblig 9 \\ inf1080}
\author{Elsie Mestl}

\begin{document}
\maketitle
\begin{flushleft}

  \section*{Oppgave 15.10:}
  \subsection*{a)}

  \[R^{\mathcal{M}} = \{\left<1 ,1 \right>, \left<1 ,2 \right>, \left<2 ,1 \right>, \left<2 ,2 \right>\}\]
  
  \subsection*{b)}

  \[R^{\mathcal{M}} = \{\left<1 ,1 \right>, \left<1 ,2 \right>\}\]

  
  \subsection*{c)}

  \[R^{\mathcal{M}} = \{\left<1 ,2 \right>, \left<2,1\right \}\]

  \subsection*{d)}
  
  \[R^{\mathcal{M}} = \{\left<1 ,2 \right>, \left<1 ,1 \right>, \left<2 ,2 \right>\}\]

  
  \section*{Oppgava 16.4:}
  
  \subsection*{a)}
  Usant, for det finnes ingen liten trekant
  
  \subsection*{b)}
  Sant, det er to små firkanter
  
  \subsection*{c)}
  Usant, x kan være liten, men være en sirkel
  
  \subsection*{d)}
  Sant, alle sirkelen er små
  
  \subsection*{e)}
  Sant, alle trekantene er store
  
  \subsection*{f)}
  Sant, la y være et av elementene i nederste rad, da finnes det ingen ellementer som er under dette.

  \section*{Oppgave 16.7:}

    Hvis utsagn 3 er sant så er alt relatert til alt. Dermed kan vi direkte trekke at de tre andre kommer som konsekvenser av dette. For da kan vil alltid trekke ut de relasjonene vi trenger.

    I tillegg har vi at utsagn 4 kommer som en konsekvens av 1 og 2. Siden alle dekker utsagnet at det finnes minst en. Kan formelt skirve:

    \begin{align*}
      \forall x \forall y (Rxy) &=> \forall x \exists y (Rxy) => \exists x \exists y (Rxy) \\
      \forall x \forall y (Rxy) &=> \exists y \forall x (Rxy) => \exists x \exists y (Rxy) \\
    \end{align*}
\end{flushleft}
\end{document}
