\documentclass[a4paper, norsk, 10pt]{article}

\usepackage[utf8]{inputenc}
\usepackage[a4]{}
\usepackage{amsmath}

\date{\today}
\title{Oblig 2 \\ inf1080}
\author{Elsie Mestl}

\begin{document}
\maketitle
\begin{flushleft}
\section*{Oppgave 4.8:}
En tautologi er et utrykk som allid er sann uavhengig av hva inputen er. \\
En kontradiksjon er et utrykk som alltid er usannt uavhengig av input

\subsection*{d}
\begin{table}[h!]
\begin{tabular}{c c || c c c }
$P$ & $Q$ & $P \rightarrow Q$ & $(P \rightarrow Q) \land \neg Q$ & $((P \rightarrow Q) \land \neg Q) \rightarrow \neg P$ \\
\hline
T & T & T & F & T \\
T & F & F & F & T \\
F & T & T & F & T \\
F & F & T & T & T 
\end{tabular}
\end{table}
Ser at hele kolonnen til høyre alltid er sann, har dermed at utrykket er en tautologi.

\subsection*{e}
\begin{align*}
\neg (P  \lor Q) \land (\neg Q \lor R ) \land (\neg R \lor P) &= \neg P \land \neg Q \land  Q \land \neg R \land R \land \neg P \\
\end{align*}

Utrykket er en kontradiksjon da Q kan ikke være både sann og usann samtidig

\subsection*{f}
\begin{align*}
(\neg (F \lor Q)) \land P = \neg F \land \neg Q \land P 
\end{align*}

Hverken en tautologi eller en kontradiksjon da utrykket er sann er når F og Q er usanne og P er sann, men usant ellers.

\section*{Oppgave 5.5:}
Bevis:
\[(P \rightarrow Q) \land (Q \rightarrow R) \rightarrow (P \rightarrow R)\]
\subsection*{a}
Direktebevis:

Setter dermed inn i en sannhetsverditabell og ser at:

\begin{table}[h!]
\begin{tabular}{c c c || c c c c c}
$P$ & $Q$ & $R$ & $A= P \rightarrow Q$ & $ B = Q \rightarrow R$ & $A \land B $ & $ P \rightarrow R$ & $A \land B \rightarrow (P \rightarrow R)$\\
\hline
T & T & T & T & T & T & T & T\\
T & T & F & T & F & F & F & T\\
T & F & T & F & T & F & T & T\\
T & F & F & F & T & F & F & T\\
F & T & T & T & T & T & T & T\\
F & T & F & T & F & F & T & T\\
F & F & T & T & T & T & T & T\\
F & F & F & T & T & T & T & T

\end{tabular}
\end{table}

Siden den siste kollonnen allitd er sann har vi at utsagnet vårt alltid vil stemme


\subsection*{c}
Motsigelsesbevis:

Anta at utrykket er usant. 
\end{flushleft}
\end{document}