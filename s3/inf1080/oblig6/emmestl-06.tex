\documentclass[a4paper, norsk, 10pt]{article}

\usepackage[utf8]{inputenc}
\usepackage[a4]{}
\usepackage{amsmath}
\usepackage{amsthm}
\usepackage{amssymb}

\date{\today}
\title{Oblig 6 \\ inf1080}
\author{Elsie Mestl}

\begin{document}
\maketitle
\begin{flushleft}


\section*{Oppgave 10.7:}

\[s: \, \{T, F\} \rightarrow \mathbb{N}\]

  $s$ er slik at \[s(P) = |\{P\}|, \quad \text{hvor } |\{P\}| \text{ er antall symboler i P}\]

La basismengden være gitt ved $X$, hvor $X$ kun har et symbol, et utsagnsvariabel. Altså: \[s(X) = 1 \]
Da er den rekursive funksjonen gitt ved:
\[s(P) = s(P') + 1 \quad \text{hvor $P'$ er $P$ med et symbol mindre.} \]
  


\section*{Oppgave 11.8:}
 
Vis at: \[6^{n} -1 \quad \text{er delelig på 5 for alle $n \in \mathbb{N}_{0}$}\]
\begin{proof} \ \\

For $n = 0$ ser vi at \[6^{0} - 1 = 1 - 1 = 0\] og \[0/5  = 0, \quad 0 \in \mathbb{N}_{0}\]

Anta sant for $n = k$.

Vis sant for $n = k+1$
\[6^{k+1} - 1= 6^{k} \cdot 6 - 1 = 6^{k}\cdot 6 -6 +5 = 6(6^{k}  -1 ) + 5\]
Siden vi etter induksjonsantagelsen har at $6^{k} -1$ er delelig på 5. Videre har vi at produktet mellom noe delelig på 5 og noe annet er delelig på 5. Og siden summen at to ledd som er delelig med 5 er selv delelig med 5. Må $6^{k+1} -1  = 6( 6^{k} -1 ) +5$ er delelig på 5. Altså er har vi at
\[6^{n} -1 \quad \text{er delelig på 5 for alle $n \in \mathbb{N}_{0}$}\]
\end{proof}
\end{flushleft}
\end{document}
