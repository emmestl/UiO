\documentclass[a4paper, norsk, 10pt]{article}

\usepackage[utf8]{inputenc}
\usepackage[a4]{}
\usepackage{amsmath}
\usepackage{amsthm}
\usepackage{amssymb}

\date{\today}
\title{Oblig 12 \\ inf1080}
\author{Elsie Mestl}

\begin{document}
\maketitle
\begin{flushleft}

  \section*{Oppgave 21.6:}

  
  \textit{La $K_{n}$ være den komplette grafen til G. $K_{n}$ har $\sum\limits_{i = 1}^{n} (i-1)$ antall kanter.}

  Fører en besvisskisse for at denne summen stemmer:

  Bevis vha iduksjon, men tenker dette ikke er en del av oppgaven så overlater dette til leseren. Forklarer heller tankegangen bak formelen.
\\ \ \\
  Tanken er at si du har en komplett graf med n noder, skal du legge til en node til og den nye grafen som lages nå også skal være komplett da vil antall kanter være lik det antall kanter i $K_{n}$ pluss en kant til alle de originale nodene til den tilsatte. Altså er antall kanter i $K_{n+1}$ lik n + antall kanter i $K_{n}$.
  Kan vha denne tankegangen finne en basis og utlede formelen gitt over.
\\ \ \\
Siden G har m kanter og vi vet at alle kanter i G ikke er kanter i $\overline{\text{G}}$ og alle ``ikke'' kanter i G er kanter i $\overline{\text{G}}$ så har vi at $\overline{\text{G}}$ har \textit{antall kanter i $K_{n}$}-m kanter. Tilsvarer dette:

\[\overline{\text{G}} = \sum\limits_{i = 1}^{n} (i-1) - m\]


\section*{Oppgave 22.10:}

  \subsection*{a)}
  $K_{m,n}$ består av n+m noder.
  
  \subsection*{b)}
  Siden $K_{m,n}$ er komplett bipartitt så vet vi at alle nodene i den mengden som består av m noder har kanter til alle nodene i den andre mengden. Da vil automatisk nodene i den andre mengden være naboer til nodene i den første. Altså trenger vi kun å se hvor mange kanter det går ut av den ene mengden for å vite det totale antall kantene i $K_{m,n}$.
  \\ \ \\
  Siden hver node i mengden m skal være nabo til alle nodene i mengden n har vi at hver node i mengden med m noder har n kanter. Altså blir det totale antall kanter i $K_{m,n}$ lik $m\cdot n$.
  
  \subsection*{c)}
  En Eulervei er definer som en vandring som er innom alle kantene i grafen en og bare en gang. En Eulerkrets er en Eulervei hvor siste og første node er like.
  \\ \ \\
  Grafen $K_{2,3}$ er en Eulerkrets (og dermed også en Eulervei). Tegn for å se.

  
\end{flushleft}
\end{document}
