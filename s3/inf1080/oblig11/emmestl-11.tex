\documentclass[a4paper, norsk, 10pt]{article}

\usepackage[utf8]{inputenc}
\usepackage[a4]{}
\usepackage{amsmath}
\usepackage{amsthm}
\usepackage{amssymb}

\date{\today}
\title{Oblig 11 \\ inf1080}
\author{Elsie Mestl}

\begin{document}
\maketitle
\begin{flushleft}

  \section*{Oppgave 19.5:}

  La \[\mathcal{M} =  \{1,2,3,4,5\}\]

  \subsection*{a)}

  Antall delmengder av $\mathcal{M}$ er (inkl den tomme mengden):
  \begin{align*}
    &5 + 5\cdot4 + 5\cdot 4 \cdot 3 + 5\cdot 4\cdot 3 \cdot 2 + 5! +1  &\text{uordet}\\*
    &\frac{5}{1} + \frac{5\cdot 4}{2!} + \frac{5\cdot 4 \cdot 3}{3!} + \frac{5 \cdot 4 \cdot 3 \cdot 2 }{4!} + \frac{5!}{5!}+1 &\text{ordnet}\\
    &=  5 + 5\cdot 2 + 5\cdot2 + 5 + 1 + 1= 32 &\text{antall delmengder av $\mathcal{M}$}    
  \end{align*}

  Må nå se på hvor mange av disse mengdene som inneholder enten 1 og/eller 2, gjør dette ved å trekke fra antaller mengder som kan bli laget av $\{3,4,5\}$. Dette inkl den tomme mengden.

  Ant delmengder av $\{3,4,5\}$:
  \begin{align*}
    &3 + \frac{3 \cdot 2}{2!} + \frac{3!}{3!} + 1= 3 + 3 + 1 +1 = 8\\
  \end{align*}

  Altså er antall delmengder av $\mathcal{M}$ som inneholder 1 og/eller 2 lik totale antall delmengder av $\mathcal{M}$ minus delmengder av $\{3,4,5\}$
  \[32 -8 = 26\]

  
  \subsection*{b)}
  En funksjon er definert ved at alle elementene i defenisjonsmengden må treffe et og bare et element i verdimengden.
  \ \\ \ \\
  Siden verdimengden ikke trenger å være den minste mengden, kan værdimengden være antallet delmengder av $\mathcal{M}$ minus den tomme mengden. Når det kommer til antall måter å fremstille en funksjon på fra $\mathcal{M}$ til en elmengde av $\mathcal{M}$ ser vi at rekkefølgen spiller en rolle. Ser altså at antall funksjoner $f: \, \mathcal{M} \rightarrow \mathcal{M}$ tilsvarer et uordent antall av delmengder av $\mathcal{M}$ altså:  325 antall funksjoner
  \ \\ \ \\ \ \\ 
  En bijektiv funksjon er når hele verdiområdet blir dekket av $f$. Ser vi på formelen over for en antall uordnede kombinasjoner. Så er vi kun interessert i det nest siste leddet i formelen (siste er Ø) for dette er når alle elementene i verdimengden, altså helle $\mathcal{M}$, blir truffet. Det tilsvarer:  $5! = 120$ forskjellige funksjoner.



  \section*{Oppgave 20.10:}
  For at $\left<G, \bullet\right>$ skal være en gruppe må:
  \begin{enumerate}
  \item $\bullet$ er assosiativt
  \item Det finnes et identitetselement for $\bullet$
  \item Alle elemter har en invers
  \end{enumerate}

  
  \subsection*{a)}
  La $\bullet = +$ og $G = \mathbb{Z}$ \\
    \ \\
  Har at oppersajonen + er assosiativ og har en identitetselement, 0.

  La $x \in G$ da er $x + x^{-1} = 0$. S ser at det inverse elementet til $x$ er $-x$ men for utenom 0 har finnes ikke $-x$ i de naturlige tallene. Altså er  $\left<G, \bullet \right>$ ikke en gruppe.  

  \subsection*{b)}

  La $\bullet = +$ og $G = \mathbb{N}$ \\
  \ \\
  Har at oppersajonen + er assosiativ og har en identitetselement, 0.

  La $x \in G$ da er $x + x^{-1} = 0$. S ser at det inverse elementet til $x$ er $-x$ altså er $\left<G, \bullet \right>$ en gruppe. Siden + er komutativt så er gruppen en abels gruppe.


  
  \subsection*{c)}
  La $\bullet = \cdot$ og $G = \mathbb{Z}$ \\

  \ \\
  Har at oppersajonen $\cdot$ er assosiativ og har en identitetselement, 1.
  La $x \in G$ da er $x \cdot x^{-1} = 1$. S ser at det inverse elementet til $x$ er $\frac{1}{x}$ men det inverset ellementet finnes kun for $x= 1$ altså er $\left<G, \bullet \right>$ ikke en gruppe.  
  
  \subsection*{d)}
  La $\bullet = +$ og $G = \mathbb{R}$ \\
    \ \\
  Har at oppersajonen + er assosiativ og har en identitetselement, 0.

  La $x \in G$ da er $x + x^{-1} = 0$. S ser at det inverse elementet til $x$ er $-x$ altså er $\left<G, \bullet \right>$ en gruppe. Siden + er komutativt så er gruppen en abels gruppe.

  
  \subsection*{e)}
  La $\bullet = \cdot$ og $G = \mathbb{R} \setminus \{0\}$ \\

  \ \\
  Har at oppersajonen $\cdot$ er assosiativ og har en identitetselement, 1.
  La $x \in G$ da er $x \cdot x^{-1} = 1$. S ser at det inverse elementet til $x$ er $\frac{1}{x}$ men det inverset ellementet finnes for alle $x \in G$ altså er $\left<G, \bullet \right>$ en gruppe.  
  


  
  \subsection*{f)}
  La $\bullet = /$ og $G = \mathbb{R} \setminus \{0\}$ \\
  
  \ \\
  Har at oppersajonen / er assosiativ men har ikke et identitetselement. Altså er $\left<G, \bullet \right>$ ikke en gruppe    
  
\end{flushleft}
\end{document}
