%Oppgaver: 1.6, 1.7 og 1.8

\documentclass[a4paper, norsk, 10pt]{article}

\usepackage[utf8]{inputenc}
\usepackage[a4]{}
\usepackage{amsmath}

\date{\today}
\title{Oblig 1 \\ inf1080}
\author{Elsie Mestl}

\begin{document}
\maketitle
\begin{flushleft}
\section*{Oppgave 1.6:}

\subsubsection*{a)} sann\\
\subsubsection*{b)} usann\\
\subsubsection*{c)} sann\\
\subsubsection*{d)} sann\\
\subsubsection*{e)} usann\\
\subsubsection*{f)} sann\\
\subsubsection*{g)} sann\\
\subsubsection*{h)} usann\\
\subsubsection*{i)} sann\\


\section*{Oppgave 1.7:}
La \[A = \{1,3,5,7,9\}, \quad B = \{0, 1,2,3,4\} \quad C = \{5,6,7,8,9\}\]
\subsubsection*{a)} \[A\setminus B = \{5,7,9\}\]
\subsubsection*{b)} \[B \setminus A = \{0,2,4\}\]
\subsubsection*{c)} \[(A\cup B) \cap C = \{5,7,9\}\]
\subsubsection*{d)} \[C \setminus (A \cup B) = \{6,8\}\]
\subsubsection*{e)} \[(A\setminus B)\setminus C = Ø\]
\subsubsection*{f)} \[(B \cup C) \setminus A = \{0,2,4,6,8\}\]

\section*{Oppgave 1.8:}
Menden \(\{1,2,3,4\}\) har delmengdene:
\[\begin{split}
  &\{1,2,3,4\}\\
&\{1,2,3\}\\
&\{1,3,4\}\\
&\{1,2,4\}\\
&\{2,3,4\}\\
&\{1,2\}\\
&\{1,3\}\\
&\{1,4\}\\
&\{2,3\}\\
&\{2,4\}\\
&\{3,4\}\\
&\{1\}\\
&\{2\}\\
&\{3\}\\
&\{4\}\\
&Ø\\
\end{split}\]
Kan også se at det er \(2^{n} = 2^{4} = 16\) løsninger. \\Ved enkelt induksjonsbevis kan vi se dette stemmer. Dermed har vi at mengden \(\{1,2,3,4,5\}\) har \( 2^{5} = 32 \) løsninger
\end{flushleft}
\end{document}
