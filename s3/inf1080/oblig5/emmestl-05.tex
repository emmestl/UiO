\documentclass[a4paper, norsk, 10pt]{article}

\usepackage[utf8]{inputenc}
\usepackage[a4]{}
\usepackage{amsmath}
\usepackage{amsthm}
\usepackage{amssymb}

\date{\today}
\title{Oblig 5 \\ inf1080}
\author{Elsie Mestl}

\begin{document}
\maketitle
\begin{flushleft}
\section*{Oppgave 8.12:}
\subsection*{a}

La $S= \mathbb{N}_{0}$ og $T = \mathbb{N}$ da er $S$ og $T$ uenderlige, tellbare mengder hvor $S \setminus T = {0}$ er endelig \\

\subsection*{b}
La $S= \mathbb{Z}$ og $T= \mathbb{N}$ da er $S$ og $T$ uenderlige, tellbare mengder hvor $S \setminus T$ ikke er endelig. Men mengden av alle negative heltall og 0 \\

\subsection*{c}
La $S = \mathbb{N}_{0}$ og la $T = \mathbb{N}\setminus \{1,2,3,4,5,6,7\}$ da er $S$ og $T$ uenderlige, tellbare mengder hvor $S \setminus T = \mathbb{N}_{0} \setminus (\mathbb{N}\setminus\{1,2,3,4,5,6,7\}) = \{0, 1, 2, 3, 4, 5, 6, 7\}$ hvor vi ser at $|S \setminus T| = |\{0, 1, 2, 3, 4, 5, 6, 7\}| = 8$


\section*{Oppgave 9.2:}
Antar at $U = \{1,2,3,a,b\}$ og la relasjonen $R$ på $U$ være gitt ved:
\[R = \{\left<2,3\right>, \left<3,2\right>, \left<1,a\right>\}\]

\subsection*{a}
Refleksiv tilsluting av R: \quad $R \cup \{\left<1,1\right>, \left<2,2\right>, \left<3,3\right>, \left<a,a\right>, \left<b,b\right> \}$


\subsection*{b}
Symetrisk tilslutning av R: \quad $R \cup \{\left<a,1\right>\}$

\subsection*{c}
Tranisitive tilslutning av R: \quad $R \cup \{\left<2,2\right>, \left<3,3\right> \}$



\section*{Oppgave 9.11:}
\subsection*{a}
\[\{a^{n}, b^{n} \, | \,n = 0,1,2,...\} = \{\Lambda, a, b, aa, bb, aaa, bbb, ... \}\]
\\
Basis: $\Lambda =  a^{0}, b^{0}$ \\
Induksjonssteg: Hvis $a^{,}$ er et multipel av $a$, og $a^{,}$ er et element i mengden. Så er også $aa^{,}$ et element i mengden. Det samme gjelder for $b$.
Tillukking: Vil her mengden oppgitt i oppgaven.


\subsection*{b}
Basis: $\Lambda = a^{0}b^{0}, \quad \Lambda \in A$ \\
Induksjonssteg: Hvis $x \in A$ så er $axb \in A$ \\
Tillukning: $\{a^{n}b^{n} \, | \, n = 0, 1, 2, ....\}$

\subsection*{c}
Basis: $\Lambda = (ab)^{0}, \quad \Lambda \in A$ \\
Induksjonssteg: Hvis $x \in A$ så er $abx \in A$ \\
Tillukning: $\{(ab)^{n} \, | \, n = 0, 1, 2, ....\}$

\end{flushleft}
\end{document}
