\documentclass[a4paper, norsk, twoside, 10pt]{article}
%trennger en pakke med norske symboler
\usepackage[utf8]{inputenc}
\usepackage[a4]{}
\usepackage[norsk]{babel}
\usepackage{amsmath}
\usepackage{amsthm}
\usepackage{amssymb}


%\usepackage{epsfig}
%\usepackage{graphicx}
%\usepackage{float}
%\usepackage{listings}
%\usepackage{color}
%\usepackage{tikz}
%\usetikzlibrary{arrows}




\date{\today}
\title{Oblig2 \\ mat1120}
\author{Elsie Mestl}

\begin{document}
\maketitle
\begin{flushleft}


  \section*{Oppgave 1}
  \subsubsection*{i)}

  Har at:
  \[
  p(t) = a_{0} + a_{1}t +  t^{2} \qquad
  C = \begin{bmatrix}
    0 & 1 \\
    -a_{0} & -a_{1} \\
  \end{bmatrix}
  \]
  \\
  \ \\


  Regner ut det karektaristiske polynomet til C
  \begin{align*}
    P(\lambda_{C}) &= \text{det}(C)\\
    &= (-\lambda)(-a_{1} - \lambda) + a_{0} \\
    &= a_{0} + \lambda a_{1} + \lambda^{2}
  \end{align*}

  Ser at \[P(\lambda_{C}) = p(t), \quad t = \lambda\]




  \subsubsection*{ii)}

  \[
  p(t) = a_{0} + a_{1}t +  a_{3}t^{2} + t^{3} \qquad
  C = \begin{bmatrix}
    0 & 1 & 0\\
    0 & 0 & 1 \\
    -a_{0} & -a_{1} & -a_{2} \\
  \end{bmatrix} \] \\
  \ \\
  Regner ut det karektaristiske polynomet til C
  \begin{align*}
    P(\lambda_{C}) &= \text{det}(C)\\
    &= -\lambda(-\lambda(-a_{2} - \lambda) + a_{1}) -a_{0}(1 - ( -\lambda \cdot 0))\\
    &= \lambda^{2}(-a_{2} - \lambda) - \lambda a_{1} - a_{0} \\
    &= -\lambda^{3} - \lambda^{2}a_{2} -\lambda a_{1} - a_{0}
  \end{align*}

  Ser at \[P(\lambda_{C}) = -p(t), \quad t = \lambda\]


  \newpage

  \section*{Oppgave 2}

  \subsection*{i)}
  Har at\[f'''(t) = 4f(t) + 4f'(t)- f''(t)\]
  Vis at $\mathbf{x}(t) = (f(t), f'(t), f''(t))$ er en løsning.
  
  \begin{align*}
    \mathbf{x}'(t) &= \mathcal{C}\mathbf{x}(t) \\
    &= \begin{bmatrix}
      0 & 1 & 0 \\
      0 & 0 & 1 \\
      4 & 4 & -1 \\
    \end{bmatrix}
    \begin{bmatrix}
      f(t) \\
      f'(t) \\
      f''(t) \\
    \end{bmatrix}
    \\
    &= \begin{bmatrix}
      0 + f'(t) + 0 \\
      0 + 0 + f''(t) \\
      4f(t) + 4f'(t) -f''(t) \\
    \end{bmatrix} \\
    &= \begin{bmatrix}
      f'(t) \\
      f''(t) \\
      f'''(t)\\
    \end{bmatrix} \\
  \end{align*}

  \subsection*{ii)}

  Vis at $f(t) = x_{1}(t)$ er en løsning for (*) når $\mathbf{x}(t) = (x_{1}(t), x_{2}(t) , x_{3}(t))$ er en løsning for (**) \\

  Finner egenverdien og egenvektorene til C. Vha matlab, og får at C har egenvektorene:

  \begin{align*}
    \mathbf{v}_{1} =
    \begin{bmatrix}
      0.25 \\
      0.5 \\
      1\\
    \end{bmatrix}
    \quad
    \mathbf{v}_{2} =
    \begin{bmatrix}
      0.25 \\
      -0.5 \\
      1\\
    \end{bmatrix}
    \quad
    \mathbf{v}_{3} =
    \begin{bmatrix}
      1 \\
      -1 \\
      1\\
    \end{bmatrix} \\
    \intertext{med respektive egenverdier:} \\
    \lambda_{1} = 2 \quad \lambda_{2} = -2 \quad \lambda_{3} = -1 \quad
  \end{align*}

  Har videre at:

  \begin{align*}
    \mathbf{x}(t) &= c_{1}\mathbf{v}_{1}e^{2t} + c_{2}\mathbf{v}_{2}e^{-2t} + c_{3}\mathbf{v}_{3}e^{-t} \\
    &= c_{1}\begin{bmatrix}
      0.25 \\
      0.5 \\
      1\\
    \end{bmatrix}e^{2t} +
    c_{2}\begin{bmatrix}
      0.25 \\
      -0.5 \\
      1\\
    \end{bmatrix}e^{-2t} +
    c_{3}\begin{bmatrix}
      1 \\
      -1 \\
      1\\
    \end{bmatrix}e^{-t} \\*
    &=
    \begin{bmatrix}
      0.25c_{1}e^{2t} + 0.25c_{2}e^{-2t} + c_{3}e^{-t} \\
      0.5c_{1}e^{2t} - 0.5c_{2}e^{-2t} - c_{3}e^{-t} \\
      c_{1}e^{2t} + c_{2}e^{-2t} + c_{3}e^{-t} \\
    \end{bmatrix}
    \intertext{Siden:}
    \mathbf{x}(t) &= (x_{1}(t), x_{2}(t) , x_{3}(t))
  \end{align*}

  Har vi:

  \begin{align*}
    x_{1}(t) &= 0.25c_{1}e^{2t} + 0.25c_{2}e^{-2t} + c_{3}e^{-t} \\
    x_{2}(t) &= 0.5c_{1}e^{2t} - 0.5c_{2}e^{-2t} - c_{3}e^{-t} \\
    x_{3}(t) &=c_{1}e^{2t} + c_{2}e^{-2t} + c_{3}e^{-t} \\
  \end{align*}

  Siden \[f(t) = x_{1}(t)  =0.25c_{1}e^{2t} + 0.25c_{2}e^{-2t} + c_{3}e^{-t}  \]
  Setter vi inn for $f(t)$ i (*) ser vi at:

  \begin{align*}
    f'''(t) +f''(t) &-4f'(t) -4f(t) \\&= -4(0.25c_{1}e^{2t} + 0.25c_{2}e^{-2t} + c_{3}e^{-t}) -4(0.5c_{1}e^{2t} - 0.5c_{2}e^{-2t} - c_{3}e^{-t} )\\ &\quad+ c_{1}e^{2t} + c_{2}e^{-2t} + c_{3}e^{-t} + 2c_{1}e^{2t}  -2c_{2}e^{-2t} - c_{3}e^{-t} \\
    &= 0 \quad \text{ved enkel algebra}
  \end{align*}

  Dermed har vi at $f(t) = x_{1}(t)$ er en løsning for (*)


  \subsection*{iii)}

  Regnet ut i oppgave 2ii)\\ Den generelle løsningen er:
  
\[\mathbf{x}(t) = c_{1}\mathbf{v}_{1}e^{2t} + c_{2}\mathbf{v}_{2}e^{-2t} + c_{3}\mathbf{v}_{3}e^{-t} \]
  

Har at \[\mathbf{x}(0) = (1 , 0 , -2)\]

Setter inn verdien for $t = 0$ og $\mathbf{x}(0)$ inn i den generelle formelen og får:

    \begin{align*}
    1 &= 0.25c_{1}e^{0} + 0.25c_{2}e^{0} + c_{3}e^{0} = 0.25c_{1} + 0.25c_{2} +c_{3} &\text{(1)}\\
    0 &= 0.5c_{1}e^{0} - 0.5c_{2}e^{0} - c_{3}e^{0} = 0.5c_{1} -0.5c_{2} - c_{3} & \text{(2)}\\
    -2 &=c_{1}e^{0} + c_{2}e^{0} + c_{3}e^{0} = c_{1} + c_{2} + c_{3} & \text{(3)}\\
    \ \\
    c_{3} &= 0.5(c_{1} - c_{2})  &\text{(2)}\\
    \ \\
    1 &= 0.25(c_{1} + c_{2}) + 0.5(c_{1} - c_{2}) = 0.75c_{1} -0.25c_{2} & \text{(1+2)}\\
    -2 &= c_{1} + c_{2} + 0.5(c_{1} - c_{2}) = 1.5 c_{1} + 0.5c_{2} & \text{(2+3)} \\
    \ \\
    c_{2} &= -4 -3c_{1} & \text{(2+3)} \\
    \ \\
    1 &= 0.75 c_{1} - 0.25(-4 -3c_{1}) = 0.75c_{1} + 1 + 0.75c_{1} = 1.5 c_{1} & \text{(1+2+3)}\\
    \ \\
    c_{1} &= \frac{2}{3} &\\
    c_{2} &= -6 &\\
    c_{3} &= \frac{10}{3} &\\
    \ \\
    \intertext{Altså er:} \\
    \mathbf{x}(t) &= \frac{2}{3}\mathbf{v}_{1}e^{2t} -6\mathbf{v}_{2}e^{-2t} + \frac{10}{3}\mathbf{v}_{3}e^{-t} \\
    &=
    \begin{bmatrix}
      \frac{1}{6}e^{2t} - \frac{3}{2}e^{-2t} + \frac{10}{3}e^{-t} \\
      \frac{1}{3}e^{2t} + 3e^{-2t} - \frac{10}{3}e^{-t} \\
      \frac{2}{3}e^{2t} - 6e^{-2t} + \frac{10}{3}e^{-t} \\
    \end{bmatrix} \\
    &=
    \begin{bmatrix}
      \frac{1}{6}e^{2t} - \frac{3}{2}e^{-2t} + \frac{10}{3}e^{-t} \\
      \frac{1}{3}e^{2t} + 3e^{-2t} - \frac{10}{3}e^{-t} \\
      \frac{2}{3}e^{2t} - 6e^{-2t} + \frac{10}{3}e^{-t} \\
    \end{bmatrix} \\
    \end{align*}
\end{flushleft}
\end{document}
