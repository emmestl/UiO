\documentclass[a4paper, norsk, twoside, 10pt]{article}
%trennger en pakke med norske symboler
\usepackage{epsfig}
\usepackage{graphicx}
\usepackage{amsmath}
\usepackage{amsthm}
\usepackage{amssymb}
\usepackage[utf8]{inputenc}
\usepackage[a4]{}
\usepackage{float}
\usepackage{listings}
\usepackage[norsk]{babel}
\usepackage{color}

\definecolor{mygreen}{RGB}{28,172,0} % color values Red, Green, Blue
\definecolor{mylilas}{RGB}{170,55,241}

\date{\today}
\title{Oblig1 \\ mat1120}
\author{Elsie Mestl}

\begin{document}
\maketitle
\begin{flushleft}

  \def\matrixP{
    \begin{bmatrix}
      1 & 0.7 & 0 & 0 & 0 \\
      0 & 0   & 0.5 & 0 & 0 \\
      0 & 0.3 & 0 & 0.65 & 0 \\
      0 & 0   & 0.5 & 0 & 0 \\
      0 & 0 & 0 & 0.35 & 1 \\
    \end{bmatrix}
  }

  \lstdefinestyle{custommat}{language=Matlab,%
    %basicstyle=\color{red},
    breaklines=true,%
    morekeywords={matlab2tikz},
    keywordstyle=\color{blue},%
    morekeywords=[2]{1}, keywordstyle=[2]{\color{black}},
    identifierstyle=\color{black},%
    stringstyle=\color{mylilas},
    commentstyle=\color{mygreen},%
    showstringspaces=false,%without this there will be a symbol in the places where there is a space
    frame = L,
    emph=[1]{for,end,break},emphstyle=[1]\color{red}, %some words to emphasise
    %emph=[2]{word1,word2}, emphstyle=[2]{style},
  }

  \section*{Oppgave 1:}
  Matlabkoden:
  \lstinputlisting[style = custommat]{opgv1.m}
  \ \\
  Gir følgende output:
  \lstinputlisting[language = matlab, frame = L]{opgv1.out}
  \ \\
  \ \\
  Der hvor det i matlab-outputen viser P = ``matrise'' tilsvarer det $P^{n}$, n gitt linjen før. Vektroen som vises
  under tilsvarer sansylighetsfordelingen etter n kjøringer. Så sansynligheten for å gå fra $s_{4}$ til $s_{2}$ er posisjon 2 i vektoren
  og presisert i teksten under.


  \section*{Oppgave 2:}

  En matriser er A regulær hvis alle elementene i $A^{n}$ for alle n, er strengt større enn 0.
  \ \\
  Har
  \def\PI{
    \begin{bmatrix}
      0 & 0.7 & 0 & 0 & 0 \\
      0 & -1   & 0.5 & 0 & 0 \\
      0 & 0.3 & -1 & 0.65 & 0 \\
      0 & 0   & 0.5 & -1 & 0 \\
      0 & 0 & 0 & 0.35 & 0 \\
    \end{bmatrix}
  }


  \begin{align*}
    P -I &= \PI
  \end{align*}
  For å finne $Nul(P - I_{5})$ løser vi likingssettet:
  \[(P-I_{5}) \vec{x} = \vec{0} \]

  Som gir den utvidede matrisen:
  \[  \begin{bmatrix}
    0 & 0.7 & 0 & 0 & 0 & 0\\
    0 & -1   & 0.5 & 0 & 0 & 0\\
    0 & 0.3 & -1 & 0.65 & 0 & 0\\
    0 & 0   & 0.5 & -1 & 0 & 0\\
    0 & 0 & 0 & 0.35 & 0 & 0\\
  \end{bmatrix}\]

  Radreduserer denne, via matlab, og får:
  \lstinputlisting[style = custommat]{opgv2.m}
  \ \\
  \lstinputlisting[language = matlab, frame = L]{opgv2.out}

  Tar hensyn til de fri variablene og får følgende likningssytemer:
  \begin{align*}
    x_{2} = 0 \\
    x_{3} = 0 \\
    x_{4} = 0 \\
  \end{align*}

  Som kan skrives som:

  \[ \begin{bmatrix}
    x_{1} \\
    x_{2} \\
    x_{3} \\
    x_{4} \\
    x_{5} \\
  \end{bmatrix}
  =
  \begin{bmatrix}
    x_{1} \\
    0 \\
    0 \\
    0 \\
    x_{5}\\
  \end{bmatrix}
    = x_{1}
  \begin{bmatrix}
    1 \\
    0 \\
    0 \\
    0 \\
    0 \\
  \end{bmatrix}
  + x_{5}
  \begin{bmatrix}
    0 \\
    0 \\
    0 \\
    0 \\
    1 \\
  \end{bmatrix}
  = x_{1}\vec{c} + x_{5}\vec{v}
  \]


Hvor $x_{1}$ og $x_{5}$ er fri variabler. Ser dermed at $\vec{c}$ og $\vec{v}$ spenner $Nul(P-I)$. De er også lineært uavhengige og dermed danner de også en basis for $Nul(P-I)$.

\end{flushleft}
\end{document}
